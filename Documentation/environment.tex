\newpage
\section{L'environnement}
\paragraph{} Chaque environnement est associé à des statistiques qui sont :
\begin{itemize}
	\item Le nombre de tour dépensé pour ce camoufler
	\item La précision des attaques
	\item Le coût en cases pour la traversé
\end{itemize}

\paragraph{} En plus de ces environnements, il existe des interfaces :
\begin{itemize}
	\item Des objets intéractifs qui coûtent un nombre de tours
	\item Des objets traversables qui coûtent un nombre de cases
\end{itemize}

\paragraph{} Certains objets interactifs peuvent ne plus être disponible pendant un moment, ou necessite une action. Par exemple un ascenseur à contre poids nécessitera de remonter le poids, une échelle pourra être détruite, il faudra donc en remettre une, etc.

\paragraph{} De plus, certains environnement pourront affecté les personnages selon leur types. Ils pourront entre autre se soigner, perdre de la vie, augmenter la porter des attaques à distances, ou encore changer la probabilité du nombre de tour au prochain jet.

\paragraph{} Certains environnement seront même innaccessible par certains type de personnages. Il faudra donc soit utiliser un objet, un sort, ou attendre que la zone s'affaiblisse.

\paragraph{} Ces zones seront donc activés n tours et désactivé k tours.

\paragraph{} Si une unité qui ne devrai pas être dans cette zone y est, elle perd un nombre de vie conséquents. Cependant, l'activation de la zone sera réduite de 1 par tour.
