\newpage
\section{Les phases de combats}

\subsection{Menu de combats}
\begin{itemize}
	\item Attaque \begin{itemize}
				      \item Attaque normal
					  \item Attaque spécial
				  \end{itemize}
	\item Interactions
	\item Utiliser un objet
	\item Se camoufler
	\item Passer son tour
\end{itemize}

\subsection{Nombre de tours} 
\paragraph{} Lors du tour du personnage, deux options s'offrent à lui : 
\begin{itemize}
	\item Avoir un nombre aléatoire disant combien de tours il peut jouer. Le résultat est pondéré par les statistiques et le type du personnage joué. 

	\item Joué que une seul fois. Cette option peut être utile dans le sens où 0 est aussi une valeur disponible dans le nombre aléatoire tiré ci dessus
\end{itemize}

\subsection{Attaques}
\paragraph{} Le nombre d'attaque disponible sur un seul joueur dépendra des statistiques de celui ci, et de sa supériorité / infériorité de classes qu'il a face au personnage d'en face.

\paragraph{} Les attaques spéciaux ne sont utilisable que une fois, et nécessite un temps de repos. Ils consomment n tour.

\paragraph{} Si un joueur nous attaque pour la première fois durant son tour, on pourra choisir deux options :
\begin{itemize}
	\item Choisir de contre attaquer pour chaque action "attaque" de l'ennemie. Le nombre de contre attaque est régis par les mêmes lois que ceux d'attaques. Chaque joueurs attaquera l'un après l'autre si son nombre de contre attaque est suffisant. Par exemple si l'ennemie peut attaquer 5 fois et moi contre attaquer 3, il y aura :
		\begin{itemize}
			\item l'ennemie attaque 2 fois
			\item Je contre attaque 1 fois
			\item l'ennemie attaque 2 fois
			\item Je contre attaque 1 fois
			\item l'ennemie attaque 1 fois
			\item Je contre attaque 1 fois
		\end{itemize}

	\item Choisir de se déplacer et de faire une des actions ci dessous :
		\begin{itemize}
			\item Se camoufler.
			\item Utiliser un objet SUR SOI.
			\item Activer un interrupteur.
			\item Détruire une partie de l'environnement, comme les échelles.
		\end{itemize}
\end{itemize}


\subsection{Déplacements}
\paragraph{} Chaque joueurs pourra se déplacer de n valeurs de cases. Les montés / descentes (via des échelles par exemple) sont aussi à prendre comme étant "des cases"

\paragraph{} Il sera impossible de se déplacer sur une unité, quelle soit allié ou ennemie. On peut cependant demander à changer de places avec elle si elle est allié, ou pousser une unité ennemie (avec taux de réussite) 

\paragraph{} De plus, on pourra se camoufler si l'on est dans le brouillard de guerre. Le camouflage fera en sorte que l'ennemie ne vous voit pas, mais consommera un nombre de n tours qui dépend des lieux. Si le nombre de tour n'est pas suffisant, il sera déduis des tours suivants.
Ce nombre de tour dépend aussi du milieux.
Certains sorts / objets permettent de détecter les unités camouflées.

\subsection{Les objets}
\paragraph{} Les objets peuvent être utilisé sur nous, alliés ou ennemies. Certains objets demandent plus de tours que d'autres (l'objet pour faire revivre par exemple consommera plus de tours que prévu).
\paragraph{} Si le nombre de tours n'est pas suffisant, on piochera sur le nombre de tours restants au prochain jet (l'option 1 en fait aussi partie).
